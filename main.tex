% !TeX spellcheck = fr_FR
\documentclass{article}
\usepackage[a4paper,pdftex]{geometry}
\usepackage[french]{babel}
\usepackage{xcolor}
\usepackage{fix-cm}
\usepackage{hyperref}
\usepackage[utf8]{inputenc} % Required for inputting international characters
\usepackage[T1]{fontenc} % Output font encoding for international characters
\usepackage{color,soul}
\usepackage{graphicx}
\usepackage{eurosym}

\setlength{\oddsidemargin}{0mm} 
\setlength{\evensidemargin}{0mm} 
\newcommand{\HRule}[1]{\hfill \rule{0.2\linewidth}{#1}} % Horizontal rule at the bottom of the page, adjust width here

\definecolor{grey}{rgb}{0.9,0.9,0.9} % Color of the box surrounding the title - these values can be changed to give the box a different color	
\definecolor{umonsgreen}{rgb}{0.56862745,0.78039216,0.63137255}
\definecolor{umonsdarkgreen}{rgb}{0.4,0.47,0.42}
\definecolor{umonsblue}{rgb}{0.23,0.85,0.98}
\definecolor{umonsdarkblue}{rgb}{0.23,0.33,0.95}
\definecolor{umonsmetalblue}{rgb}{0.21,0.67,0.61}
\definecolor{umonsred}{RGB}{168, 0, 57}
\definecolor{umonsgray}{RGB}{150, 150, 150}
\definecolor{umonsturquoise}{RGB}{0, 171, 204}


\setul{0.3ex}{0.2ex}
%\setulcolor{umonsred}
\newcommand{\umonsU}{\textbf{\color{umonsgray}\setulcolor{umonsred}\ul{U}}}


\usepackage[backend=biber,style=numeric,natbib=true]{biblatex} % Use the bibtex backend with the authoryear citation style (which resembles APA)

\addbibresource{sources.bib} % The filename of the bibliography

\usepackage[autostyle=true]{csquotes} % Required to generate language-dependent quotes in the bibliography

\begin{document}

\thispagestyle{empty}

%----------------------------------------------------------------------------------------
%	TITLE SECTION
%----------------------------------------------------------------------------------------

\colorbox{umonsmetalblue}{
	\parbox[t]{1.0\linewidth}{
		\centering \fontsize{40pt}{80pt}\selectfont % The first argument for fontsize is the font size of the text and the second is the line spacing - you may need to play with these for your particular title
		\vspace*{0.7cm} % Space between the start of the title and the top of the grey box
		
		\hfill Questions théoriques \\
		\hfill Entrepreneuriat\\
		\hfill 2016-2017\par
		
		\vspace*{0.7cm} % Space between the end of the title and the bottom of the grey box
	}
}

%----------------------------------------------------------------------------------------

\vfill % Space between the title box and author information

%----------------------------------------------------------------------------------------
%	AUTHOR NAME AND INFORMATION SECTION
%----------------------------------------------------------------------------------------

{\centering \Large 
	\hfill Plein de gens \\
	\hfill Plein de sections \\
	\hfill {\fontfamily{ptm}\selectfont \umonsU \textcolor{umonsred}{MONS}} \\
	\hfill
	\href{mailto:t-as-vraiment-cliqué}{Plein d'adresses mail} \\
	
	\HRule{1pt}} % Horizontal line, thickness changed here

%----------------------------------------------------------------------------------------

\clearpage % Whitespace to the end of the page
\setcounter{tocdepth}{1}
\tableofcontents
\clearpage % Whitespace to the end of the page

%todo: choppez une des sections (les "todos" suivants), certains ont 2 questions, d'autres 3 mais a priori certaines questions sont plus faciles donc ça devrait être +/- réparti équitablement.

%todo: 1
\section{Quels sont les points (contenu) à aborder lors du pitch et que doit contenir un executive summary ?}
\section{Quelle est la structure classique d’un plan d’affaires ?}
\section{Quels sont les 2 différents aspects à traiter pour le marché, et à quelles questions doit-on répondre?}
%todo: 2
\section{Définissez Marché, segment de marché et niche.}
\section{Que faut-il comprendre / analyser dans l’offre des concurrents et pourquoi ?}
\section{Quels sont les objectifs du plan financier ?}
%todo: 3 (Il serait bien d'en détailler plusieurs pour ne pas se retrouver tous à epliquer le même ;-))
\section{Quelles sont les différentes mesures de la rentabilité des investissements ? Détaillez-en une.}
\section{Quelle est la principale mesure de la rentabilité de l’exploitation ? Quelles en sont les limites ?}
%todo: 4
\section{Définissez le BFR et pourquoi faut-il surveiller son évolution ?}
\section{Citez des actions permettant d’améliorer la position de trésorerie d’une entreprise ?}
%todo: 5
\section{Expliquez les trois concepts essentiels analysés lors d’un diagnostic financier.}
\section{Définissez la notion de Valeur Ajoutée.}
\section{Qu’est ce que le cash flow d’une entreprise ou pourquoi est-ce vital ?}

\printbibliography

\end{document}