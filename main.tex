% !TeX spellcheck = fr_FR
\documentclass{article}
\usepackage[a4paper,pdftex]{geometry}
\usepackage[french]{babel}
\usepackage{xcolor}
\usepackage{fix-cm}
\usepackage{hyperref}
\usepackage[utf8]{inputenc} % Required for inputting international characters
\usepackage[T1]{fontenc} % Output font encoding for international characters
\usepackage{color,soul}
\usepackage{graphicx}
\usepackage{eurosym}
\usepackage{float}

\setlength{\oddsidemargin}{0mm} 
\setlength{\evensidemargin}{0mm} 
\newcommand{\HRule}[1]{\hfill \rule{0.2\linewidth}{#1}} % Horizontal rule at the bottom of the page, adjust width here

\definecolor{grey}{rgb}{0.9,0.9,0.9} % Color of the box surrounding the title - these values can be changed to give the box a different color	
\definecolor{umonsgreen}{rgb}{0.56862745,0.78039216,0.63137255}
\definecolor{umonsdarkgreen}{rgb}{0.4,0.47,0.42}
\definecolor{umonsblue}{rgb}{0.23,0.85,0.98}
\definecolor{umonsdarkblue}{rgb}{0.23,0.33,0.95}
\definecolor{umonsmetalblue}{rgb}{0.21,0.67,0.61}
\definecolor{umonsred}{RGB}{168, 0, 57}
\definecolor{umonsgray}{RGB}{150, 150, 150}
\definecolor{umonsturquoise}{RGB}{0, 171, 204}


\setul{0.3ex}{0.2ex}
%\setulcolor{umonsred}
\newcommand{\umonsU}{\textbf{\color{umonsgray}\setulcolor{umonsred}\ul{U}}}


\usepackage[backend=biber,style=numeric,natbib=true]{biblatex} % Use the bibtex backend with the authoryear citation style (which resembles APA)

\addbibresource{sources.bib} % The filename of the bibliography

\usepackage[autostyle=true]{csquotes} % Required to generate language-dependent quotes in the bibliography

\begin{document}

\thispagestyle{empty}

%----------------------------------------------------------------------------------------
%	TITLE SECTION
%----------------------------------------------------------------------------------------

\colorbox{umonsmetalblue}{
	\parbox[t]{1.0\linewidth}{
		\centering \fontsize{40pt}{80pt}\selectfont % The first argument for fontsize is the font size of the text and the second is the line spacing - you may need to play with these for your particular title
		\vspace*{0.7cm} % Space between the start of the title and the top of the grey box
		
		\hfill Questions théoriques \\
		\hfill Entrepreneuriat\\
		\hfill 2016-2017\par
		
		\vspace*{0.7cm} % Space between the end of the title and the bottom of the grey box
	}
}

%----------------------------------------------------------------------------------------

\vfill % Space between the title box and author information

%----------------------------------------------------------------------------------------
%	AUTHOR NAME AND INFORMATION SECTION
%----------------------------------------------------------------------------------------

{\centering \Large 
	\hfill Plein de gens \\
	\hfill Plein de sections \\
	\hfill {\fontfamily{ptm}\selectfont \umonsU \textcolor{umonsred}{MONS}} \\
	\hfill
	\href{mailto:t-as-vraiment-cliqué@pigeon.you}{Plein d'adresses mail} \\
	
	\HRule{1pt}} % Horizontal line, thickness changed here

%----------------------------------------------------------------------------------------

\clearpage % Whitespace to the end of the page
\setcounter{tocdepth}{1}
\tableofcontents
\clearpage % Whitespace to the end of the page

%todo: choppez une des sections (les "todos" suivants), certains ont 2 questions, d'autres 3 mais a priori certaines questions sont plus faciles donc ça devrait être +/- réparti équitablement. Petit conseil: mettre des dessins accompagnés d'explications(un dessin est plus facile à retenir qu'un texte). Bon site pour faire des schémas: https://www.draw.io/

%todo EMILIEN PERETTI
\section{Quels sont les points (contenu) à aborder lors du pitch et que doit contenir un executive summary ?}
Le but du pitch est de donner envie à l'interlocuteur de découvrir notre "Executive
Summary". Celui-ci doit comporter la réponse aux questions suivantes: 
\begin{itemize}
\item Qui vous êtes et comment vous contacter ?
\item Quel est le problème que le produit / service résout ?
\item Quel est la valeur ou le bénéfice qu’il apporte ?
\item Pourquoi est-ce important et excitant?
\item Qu’avez-vous déjà accompli et de quoi avez-vous besoin pour réussir ?   
\end{itemize}
L'Executive Summary,quant à lui, a pour but d'obtenir une invitation pour présenter le projet et donner envie de lire le plan financier. Il doit contenir les points suivants:
\begin{itemize}
\item Origine du projet
\item Résumé de la solution proposée
\item Marché
\item Stratégie
\item La société et l’équipe
\item Concurrence et avantage compétitif
\item Synthèse stratégie et plan d’actions
\item Principaux chiffres (chiffre d'affaire, bénéfice, cashflow, investissements et besoins financiers)
\end{itemize}
\section{Quelle est la structure classique d’un plan d’affaires ?}
Le structure classique d'un plan d'affaires est la suivante:
\begin{enumerate}
\item Définition du projet et de l’historique
\item L’équipe
\item La solution «produit/service»
\item Le(s) marché(s)
\item La concurrence
\item La vision – La stratégie et les objectifs – Le  plan d’action
\item Les besoins financiers (plan financier)
\item La proposition d’investissement
\end{enumerate}

\section{Quels sont les 2 différents aspects à traiter pour le marché, et à quelles questions doit-on répondre?}

%%%%%%%%%%%%%%%%%%%%%%%%%%%%%%%%%%%%%%%%%%%%%%%%%%%%%%%%%%%%%%%%%%%%%%%
%todo: Duncan
\section{Définissez Marché, segment de marché et niche.}
\begin{tabular}{|rcl|}
\hline
&&\\
&& \multicolumn{1}{p{.8\textwidth}|}{Définition simple: "Ensemble de sous-marchés constitués de clients présentant des caractéristiques homogènes (segments de marché)"}\\
Marché & : & \multicolumn{1}{p{.8\textwidth}|}{Simplement, le marché est le sur-ensemble de plusieurs segments de marché. Par exemple, le marché de l'électroménager regroupe les segments de marché des lave-vaisselle, des fours, des machines à laver, etc... Chacun de ces segments regoupe un type de clients précis ("les clients qui veulent une machine à laver", "les clients qui veulent un four/ une cuisine équipée", "les clients qui...)}\\
&&\\
\hline
&&\\
&& \multicolumn{1}{p{.8\textwidth}|}{Définition simple: "Regroupement homogène de personnes (selectionnées sur de nombreux critères) sur lesquelles il est possible d'effectuer des actions marketing de différenciations." - "= stratégie adaptée"}\\
Segment de Marché & : & \multicolumn{1}{p{.8\textwidth}|}{En gros, on peut segmenter le marché global en plus petits groupes sur lesquels la publicité (marketing) va pouvoir effectuer des actions ciblées. Par exemple, un grand nombre de gens aiment le coca cola mais en segmentant le marché des consommateurs par âge, coca peut par exemple faire ses campagnes pub "fun et fête" en visant le segment 15-30 ans. Ou la pub de Noël qui est plutôt pour les enfants (magique, papa noël, toussah,...)}\\
&&\\
\hline
&&\\
Niche & : & \multicolumn{1}{p{.8\textwidth}|}{Petite maison pour chien généralement placée dans le fond du jardin.}\\
&&\\
\hline
&&\\
				 &   & \multicolumn{1}{p{.8\textwidth}|}{Un marché de niche est un marché très étroit correspondant à un produit ou service très spécialisé (ciblé).}\\
Marché de Niche & : & \multicolumn{1}{p{.8\textwidth}|}{Le fait de viser un marché de niche permet souvent d’être confronté à une concurrence moins forte et à un potentiel de marges plus élevées, mais les volumes de ventes potentiels sont naturellement plus faibles et limités.}\\
				 &  & \multicolumn{1}{p{.8\textwidth}|}{Le caractère de niche de marché est une notion très relative selon les contextes. Un marché de niche peut être de quelques centaines de milliers d’euros ou de quelques millions, voire dizaines de millions d’euros, si on se trouve dans le domaine du marché automobile.}\\
				 &  & \multicolumn{1}{p{.8\textwidth}|}{En gros, Un marché de niche c'est donc un marché qui va intéresser des gens cherchant un produit très particulier. Une entreprise se lançant actuellement dans la vente de fusées pour aller sur la lune VA se trouver sur un marché de Niche (enfin je pense pas que beaucoup de gens veulent acheter des fusées lunaires) Alors vu que c'est un marché avec peu de clients ben d'un côté on va pas avoir beaucoup de ventes mais de l'autre on peut fixer le prix et/car la concurrence est faible.}\\
&&\\
\hline
\end{tabular}
\section{Que faut-il comprendre / analyser dans l’offre des concurrents et pourquoi ?}
Tout d'abord, l'analyse de la concurrence est une \textbf{étape clé d'un projet d'entreprise}. C'est une étape importante qui est souvent sous-estimée.

Pour ce faire, il faut analyser la relation entre les entreprises concurrentes et regardant \textbf{la rivalité} existante entre ces concurrents. Il faut regarder aussi les caractèristiques de notre marché en mesurant la \textbf{menace des entrants potentiels}, les \textbf{pouvoir de négociation des clients et des fournisseurs} et les \textbf{menaces des produits substituts}.

\begin{figure}[H]
	\centering
	\includegraphics[width=14cm]{etudeConcurrence1.png}
\end{figure}
\begin{figure}[H]
	\centering
	\includegraphics[width=14cm]{etudeConcurrence2.png}
\end{figure}
\begin{figure}[H]
	\centering
	\includegraphics[width=14cm]{etudeConcurrence3.png}
\end{figure}
\begin{figure}[H]
	\centering
	\includegraphics[width=14cm]{etudeConcurrence4.png}
\end{figure}
\begin{figure}[H]
	\centering
	\includegraphics[width=14cm]{etudeConcurrence5.png}
\end{figure}
\begin{figure}[H]
	\centering
	\includegraphics[width=14cm]{etudeConcurrence6.png}
\end{figure}

On peut aussi tenter de comprendre et analyser leur avantages et désavantages par rapport à notre entreprise, ceci peut se faire via \textbf{le canevas stratégique} [\ref{fig:canevasStrategique}]. Celui-ci permet de positionner la concurrence par rapport à son entreprise sur une série de critères choisis. Ces critères doivent donc être bien choisis si on veut que l'analyse soit bien faite.
\begin{figure}[H]
	\centering
	\includegraphics[width=12cm]{canevasStrategique.png}
	\caption{Canevas stratégique}
	\label{fig:canevasStrategique}
\end{figure}

Cette analyse nous permet de voir sur quels critères notre entreprise est meilleure (océan bleu) et sur quels critères elle doit mieux faire ou sur quels critères il ne faut pas miser car "les autres font mieux" (océan rouge). Le positionnement en océan bleu permet à l'entrerpise d'avoir un peu d'espace stratégique et un peu d'air pour se développer sans réelle conccurence.\\

Au delà de ça, cette analyse permet de construire sa matrice ERAC [\ref{fig:ERAC}] qui permet de renforcer la valeur de son offre mais ce n'est pas l'objectif de la question.
\begin{figure}[H]
	\centering
	\includegraphics[width=14cm]{ERAC.png}
	\caption{Les différents éléments de la matrice ERAC}
	\label{fig:ERAC}
\end{figure}

\section{Quels sont les objectifs du plan financier ?} %TODO
Le plan financier a plusieurs objectifs sous deux aspects.
	\subsection{Aspects Internes}
		\paragraph{Valider la faisabilité financière du projet}~\\
			L'établissement du plan financier demande de calculer les couts et revenus du produit et permet ainsi de se rendre compte de la viabilité du projet au niveau financier.
			
		\paragraph{Choisir entre diverses options stratégiques}~\\
			Etant donné que l'on calcule les couts selon les différentes stratégies, on peut évaluer chacune d'entre elles et choisir la plus profitable/la moins risquée.
			
		\paragraph{Mesurer la rentabilité attendue du projet}~\\
			Le plan financier vise aussi à un calcul de la propre rentabilité de l'entreprise c.f.:\ref{mesureRentabilite}.
			
		\paragraph{Déterminer les besoins financiers}~\\
			La réalisation du calcul des couts permet de voir quel investissement au départ est nécessaire pour que l'entreprise survive.
		\paragraph{Communiquer sur les objectifs}~\\
			En établissant le plan financier, on va aussi prévoir les bénéfise à 3,5 et/ou 10 ans. cela permet ainsi à l'entreprise de voir quels sont ses objectifs et permet aussi un suivi de ceux-ci.
		
	\subsection{Aspects Externes}
		\paragraph{Communiquer avec les partenaires de l’entreprise}~\\
			Le plan financier permet d'afficher de manière "publique" les résultats attendus, les couts et les dépenses de l'entreprise. Ceci permet une bonne communication entre l'entreprise et ses fournisseurs par exemple.
		\paragraph{Être un outil de support à la négociation avec les partenaires financiers pressentis}~\\
			La banque sera plutot d'accord de faire un pret à une entreprise qui lui présente un plan financier clair et annoncant des beaux chiffres :D 
			

%%%%%%%%%%%%%%%%%%%%%%%%%%%%%%%%%%%%%%%%%%%%%%%%%%%%%%%%%%%%%%%%%%%%%%%
%todo: Adrien
\section{Quelles sont les différentes mesures de la rentabilité des investissements ? Détaillez-en une.}\label{mesureRentabilite}
\subsection{VAN}
\begin{figure}[H]
	\centering
	\includegraphics[width=14cm]{van.jpg}
\end{figure}

\paragraph{Un exemple pratique:}
Une entreprise envisage l’acquisition d’une machine d’une valeur de 10.000\euro{}, utilisable et amortissable en linéaire sur 5 ans.
Au bout des 5 ans, elle pourra encore être revendue à 1.000\euro{}.
Cette machine permettrait d’améliorer le chiffre d’affaires de 7.000\euro{} par an et augmenterait les coûts de 2.000\euro{} par an. Le taux d’ actualisation retenu est de
10\%. 30\% d’impôts. Cet investissement peut-il être envisagé ? (voir table \ref{van})

\begin{table}[H]
	%\begin{center}
\begin{tabular}{l c r}
	chiffre d’ affaires supplémentaire &=&	7.000 \euro{}\\
	- coût supplémentaire	&=&	-2.000 \euro{} \\
	\hline
	résultat brut généré	&=&	5.000 	\euro{}\\
	- amortissements	&=&	-2.000 	\euro{}\\
	\hline
	résultat d’	exploitation	&=&	3.000 \euro{}\\
	- impôts (30\%)	&=&	-900 	\euro{}\\
	\hline
	\color{red}résultat net	&\color{red}=&	\color{red}2.100 	\euro{}\\
	+ amortissements	&=&	+2.000 	\euro{}\\
	\hline
	\color{red}F	&\color{red}=&	\color{red}4.100 	\euro{}\\
\end{tabular}
%\end{center}
\caption{\label{van} Calcul du Flux de Revenus (F) hors économie fiscale}
\end{table}

Donc, l’investissement rapporte :
\begin{itemize}
	\item en année 1 : $F_1$ = -5.900 \euro{}
	\item en année 2 : $F_2$	= 	4.100 \euro{}
	\item en année 3 : $F_3$	= 	4.100 \euro{}
	\item en année 4 :	$F_4$	= 	4.100 \euro{}
	\item en année 5 : $F_5$	= 	4.100 	\euro{}
	\item en année 5 : la valeur résiduelle de la machine, soit, $1000 (1-0,30) = 700$ \euro{}
\end{itemize}
\begin{figure}[H]
	%\centering
	\includegraphics[width=10cm]{van-rentabilite.jpg}
\end{figure}

\subsection{TRI}

\begin{figure}[H]
	\centering
	\includegraphics[width=14cm]{tri.jpg}
\end{figure}

Sur l'exemple:

\begin{figure}[H]
	%\centering
	\includegraphics[width=10cm]{tri-exemple.jpg}
\end{figure}

Notes : 
Une tentative de graphe est dispo dans les slides sur la relation entre TRI et VAN mais ne veut absolument rien dire (si ce n'est mettre en évidence le fait que VAN=\euro{} et TRI=\%).
Les slides présentent aussi la méthode PRA pour évaluer le temps nécessaire pour récupérer les montants investis mais précise que ça ne mesure pas la rentabilité du projet (malgré que ça se trouve dans la section rentabilité des investissements).

\section{Quelle est la principale mesure de la rentabilité de l’exploitation ? Quelles en sont les limites ?}

\subsection{BEP}
Le seul de rentabilité (break-even point, BEP). Il répond à la question \textquote{quelle doit être la quantité à produire et à vendre de telle manière que les revenus (c’est-à-dire le chiffre d’affaires) couvrent l’ensemble des coûts?}. Au moment où on l'atteint, il n'y a ni gain, ni perte.

\begin{figure}[H]
	%\centering
	\includegraphics[width=10cm]{bep.jpg}
\end{figure}

\begin{figure}[H]
	\centering
	\includegraphics[width=10cm]{bep-graphe.jpg}
\end{figure}

\subsection{Ses limites}
Les limites de l’approche «seuil de rentabilité» :
\begin{itemize}
	\item Les prix de vente et les coûts variables unitaires sont-ils constants dans 
	le temps et indépendants du volume?
	\item Comment appliquer la méthode quand on fait face à des produits 
	multiples, complémentaires ou substituables?
	\item Les frais fixes sont-ils constants dans le temps et réellement
	indépendants du volume de production/vente?
\end{itemize}
~\\
$\rightarrow$ Notion essentiellement de court terme. Elle suppose:

\begin{itemize}
	\item des investissements donnés,
	\item Un product mix donné,
	\item De faibles variations en termes de volume.
\end{itemize}

%%%%%%%%%%%%%%%%%%%%%%%%%%%%%%%%%%%%%%%%%%%%%%%%%%%%%%%%%%%%%%%%%%%%%%%
%todo: Cassandra
\section{Définissez le BFR et pourquoi faut-il surveiller son évolution ?}

\subsection{Définition du fonds de roulement net}
Définissons d'abord le fonds de roulement net (FRN). On a
\begin{center}
	FRN = Capitaux permanents - actifs fixes
\end{center}

\includegraphics[width=11cm]{c6.png}

Pour produire des biens/services, l'entreprise a besoin "d'outils de production":
\begin{enumerate}
	\item Brevets, licenses,...
	\item Terrains
	\item Bâtiments
	\item Installation, machines, outillages
	\item Mobilier et matériel roulant
	\item Immobilisation en location financement
	\item ... 
\end{enumerate}

L'entreprise a donc besoin de capitaux permanents pour financer l'acquisition de ces "outils". Le FRN se calcule alors de cette manière:

\begin{center}
	FRN = capitaux propres + dettes LT - actifs fixes.
\end{center}

\includegraphics[width=10cm]{c7.png}

Sont compris dans les fonds propres: 
\begin{enumerate}
	\item Le capital
	\item Les primes d'émission (mécanisme financier permettant d'apporter aux anciennes actions une valeur supérieure à leur valeur nominale lors d'une augmentation de capital)
	\item Les plus-values de réévaluation
	\item Les Réserves
	\item Les subsides en capital
	\item Les provisions et impôts différés
\end{enumerate}

Sont compris dans les actifs fixes:
\begin{enumerate}
	\item Les frais d'établissement
	\item Les immobilisations incorporelles
	\item Les immobilisations corporelles
	\begin{enumerate}
		\item Terrains et constructions
		\item Installations, machines et outillafe
		\item Mobilier et matériel roulant
		\item Location financement et droits similaires
	\end{enumerate}
	\item Les immobilisations financières
	\item Les créances à LT ( $>$ 1 an)
\end{enumerate}

\subsection{Définition du besoin en fonds de roulement}
Le BFR, quant à lui, est le besoin en fonds de roulement. 

\includegraphics[width=15cm]{c1.png}

Sont compris dans les actifs circulants:
\begin{enumerate}
	\item Les stocks et commandes en cours d'éxécution
	\item Les créances à un an au plus
	\begin{enumerate}
		\item Les dettes à plus d'un an échéant dans l'année
		\item Les dettes dont le terme contractuel est d'un an au plus
		\begin{enumerate}
			\item Dettes financières
			\item Dettes commerciales
			\item Acomptes reçus sur commande
			\item Dettes fiscales, salariales ou sociales
			\item Autres
		\end{enumerate}
	\end{enumerate}
	\item Les placements de trésorerie
	\item Les valeurs disponibles (valeurs qu'une entreprise peut utiliser immédiatement pour effectuer des règlements)
\end{enumerate}

\includegraphics[width=15 cm]{c4.png}

Le BFR d'exploitation est ce dont l'entreprise à besoin pour faire fonctionner son cycle d'exploitation. Il se calcule de cette manière:

\begin{center}
	BFR expl = Stocks + créances commerciales - Dettes des fournisseurs.
\end{center}

\includegraphics[width=15cm]{c2.png}

\subsection{Comment estimer le BFR}

\includegraphics[width=8cm]{c5.png}

\subsection{Pourquoi faut-il surveiller son évolution ?}
La différence entre le FRN et le BFR est appelée la Trésorerie nette (TN).
\begin{center}
	TN = FRN - BFR = placements de trésorerie + valeurs disponibles - dettes financières CT
\end{center}
Si les moyens durables de financement de l'activité courante (FRN) sont supérieurs aux besoins de liquidités (BFR), alors l'entreprise disposera de liquidités:


\includegraphics[width=13cm]{c9.png}

Par contre, si le fonds de roulement net est insuffisant pour couvrir le besoin de fonds de roulement, alors l'entreprise devra avoir recours à un découvert ou un crédit de caisse:

\includegraphics[width=10cm]{c8.png}

L'évolution du BFR:

\begin{enumerate}
	\item L'augmentation du volume d'affaires entraine une augmentation du besoin de fonds de roulement
	\item L'entreprise doit, dès lors, vérifier si sa croissance ne va  pas mettre à mal sa liquidité
	\item Cas de la croissance sauvage avec aggravation du déficit de trésorerie
\end{enumerate}

\includegraphics[width=12 cm]{c3.png}

\section{Citez des actions permettant d'améliorer la position de trésorerie d'une entreprise ?}

Voir définition de trésorerie question 9.

Pour améliorer la trésorerie de l'entreprise, deux types d'actions peuvent être menées:

Sur les éléments à court terme:
\begin{enumerate}
	\item Raccourcissement des délais de paiement des cleints
	\item Pratique d'acomptes
	\item Accélération du cycle administratif de facturation
	\item Allongement des délais de paiement aux fournisseurs
	\item Diminution des stocks
\end{enumerate}
Sur les éléments à long terme:
\begin{enumerate}
	\item Augmentation des capitaux permanents par:
	\begin{enumerate}
		\item des emprunts à long teme,
		\item une recapitalisation de l'entreprise
	\end{enumerate}
	\item Diminution des actifs fixes par:
	\begin{enumerate}
		\item revente d'actifs financiers
		\item désinvestissement (judicieux?).
	\end{enumerate}
\end{enumerate}


%%%%%%%%%%%%%%%%%%%%%%%%%%%%%%%%%%%%%%%%%%%%%%%%%%%%%%%%%%%%%%%%%%%%%%%
%todo: 5
\section{Expliquez les trois concepts essentiels analysés lors d’un diagnostic financier.}
\section{Définissez la notion de Valeur Ajoutée.}
\section{Qu’est ce que le cash flow d’une entreprise ou pourquoi est-ce vital ?}

\printbibliography

\end{document}